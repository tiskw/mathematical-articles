% TeX source
%
% Author: Tetsuya Ishikawa <tiskw111@gmail.com>
% Date  : October 13, 2021
%%%%%%%%%%%%%%%%%%%%%%%%%%%%%%%%%%%%%%%%%%%%%%%%%%%%% SOURCE START %%%%%%%%%%%%%%%%%%%%%%%%%%%%%%%%%%%%%%%%%%%%%%%%%%%%%

\documentclass[uplatex, twocolumn, a4paper, 10pt, dvipdfmx]{jsarticle}

\usepackage[dvipdfm,width=190truemm,height=270truemm]{geometry}
\usepackage{amsmath,amssymb}
\usepackage{lmodern}
\usepackage{helvet}
\usepackage[sourcehan, unicode]{pxchfon}
\usepackage[dvipdfmx]{graphicx}
\usepackage{thmbox}
\usepackage{csquotes}
\usepackage{algorithm2e}
\usepackage{algorithmic}
\usepackage{fancyhdr}
\usepackage{xcolor}

\pagestyle{fancy}
\lhead{}
\rhead{}
\cfoot{-- {\thepage} --}
\rfoot{\small\tt\textcolor{lightgray}{T.~Ishikawa}}
\renewcommand{\headrulewidth}{0pt}
\renewcommand{\footrulewidth}{0pt}

\allowdisplaybreaks

\newcommand{\bs}[1]{\boldsymbol{#1}}
\newcommand{\mbar}{\hspace{1pt}\bigm|\hspace{1pt}}
\newcommand{\tran}{^{\mkern-1.5mu\mathsf{T}}}
\DeclareMathOperator*{\sgn}{sgn}
\DeclareMathOperator*{\Sgn}{Sgn}
\DeclareMathOperator*{\argmin}{argmin}

\newtheorem[M]{theorem}{定理}[section]
\newtheorem[M]{definition}[theorem]{定義}
\newtheorem[M]{lemma}[theorem]{補題}

\renewenvironment{proof}%
{\textsf{証明}:}{\hfill$\blacksquare$\vspace*{8pt}}

\RestyleAlgo{ruled}
\SetKwComment{Comment}{/* }{ */}

\newcommand{\titlebar}{
\setlength\unitlength{1truemm}
\vspace*{-19pt}
\begin{picture}(88,0)(0,0)
\put(-4,0){\line(1,0){93}}
\put(89,0){\circle*{1}}
\end{picture}}

\begin{document}

\title{\textsf{\bfseries ガウス過程モデルと Random Fourier Features}}
\author{石川 徹也 \quad \texttt{<tiskw111@gmail.com>}}
\date{2021年10月13日}
\maketitle
\thispagestyle{fancy}

\section*{はじめに}
\titlebar
% TeX source


現代の確率論は,その歴史的過程により,測度論を用いて高度に形式化されています.
形式化されること自体は,厳密さを尊ぶ現代数学の理念に叶うものであり,大いに歓迎されるべきなのですが,
その一方で,残念なことに,過度な形式化は概念の本質を覆い隠してしまうことがあります.
その最たるもののひとつが,確率密度関数です.

今,試みに確率の書籍を開いて確率密度関数の定義を見てみましょう.
伊藤~\cite{Ito2004} には次のように定義されています.

\begin{theorem}{確率密度関数の定義}
    \label{thm:prob_density_func_formal}
    確率空間 $(\Omega, \mathcal{F}, P)$ を考える.
    Radon–Nikodym の定理より,任意の事象 $E \in \mathcal{F}$ に対し
    \begin{equation}
        P(E) = \int_E p(\omega) \mathrm{d} \mu(\omega),
    \end{equation}
    をみたす関数 $p(\omega)$ が存在する(Radon–Nikodym 微分).
    これを確率測度 $P$ の確率密度と呼ぶ.これは微分形
    \begin{equation}
        p(\omega) \triangleq \frac{\mathrm{d} P}{\mathrm{d} \mu}(\omega),
    \end{equation}
    で表されることもある.
\end{theorem}

上記は任意の確率空間に対して適用できるよう抽象化された表現であり,さすがにちょっと分かりにくいので,
実数体に値を取る確率変数 $X$ に限定した表現に直してみましょう.
詳細は省略しますが,実数値確率変数 $X$ の場合,上記の確率密度関数は

\begin{equation}
    p_X(x) \triangleq \frac{\mathrm{d} P(X \leq x)}{\mathrm{d} x},
    \label{eqn:def_pdf_1}
\end{equation}

と書くことができます.累積分布関数 $f_X(x) \triangleq P(X \leq x)$ を導入すれば

\begin{equation}
    p_X(x) \triangleq \frac{\mathrm{d} f_X}{\mathrm{d} x},
    \label{eqn:def_pdf_2}
\end{equation}

と表現するのと同じことです(何となく抽象的な定義の微分形に似ていることが分かるかと思います).

ある程度確率論に慣れた経験者にとって,式 (\ref{eqn:def_pdf_1}) や 式 (\ref{eqn:def_pdf_2}) は
「なるほど上手く定義したものだ」と思える代物だと思います.
ですが,そもそも確率密度関数の意味を知らない確率論の初学者に
式 (\ref{eqn:def_pdf_1}) や 式 (\ref{eqn:def_pdf_2}) を見せたところで

\begin{center}
    \gt 「で,確率密度関数って結局何なの?」
\end{center}

という疑問が浮かぶだけでありましょう.このような疑問が浮かぶのも至極当然です.
式 (\ref{eqn:def_pdf_1}) や 式 (\ref{eqn:def_pdf_2}) は,形式的に数学を構築していくには大変都合の良い定義ですが,
確率密度関数とは何かという本質的な問いには全く答えていない式だからです.これが過度な形式化の弊害なのです.

本文書の目的は「確率密度関数とは何か?」という疑問に対し,定義式を見せる以外の手段で回答を試みることにあります.
もし本文書を読み終えて式 (\ref{eqn:def_pdf_1}) を見直したとき,少しでも「なるほど上手い定義式だ」と感じて
頂けたのであれば,本文書は成功だと言えるでしょう.


% vim: expandtab tabstop=4 shiftwidth=4 fdm=marker


\section{ガウス過程モデル再訪}
\titlebar
% TeX source
%
% Author: Tetsuya Ishikawa <tiskw111@gmail.com>
% Date  : October 13, 2021
%%%%%%%%%%%%%%%%%%%%%%%%%%%%%%%%%%%%%%%%%%%%%%%%%%%%% SOURCE START %%%%%%%%%%%%%%%%%%%%%%%%%%%%%%%%%%%%%%%%%%%%%%%%%%%%%

本節ではガウス過程モデルの概要について述べます.
残念ながら本文書ではガウス過程モデルの定式化や導出などの詳細は扱いませんので,
詳細にご興味のある読者は Rasmussen~\cite{Rasmussen2006} あるいは赤穂~\cite{Akaho2018}をご参照下さい.

学習データを$\mathcal{D} = \{ (\bs{x}_n, y_n) \}_{n=1}^{N}$,
ラベルの観測誤差の標準偏差を$\sigma \in \mathbb{R}^{+}$とします.
ただし$\bs{x}_n \in \mathbb{R}^M$, $y_n \in \mathbb{R}$とします.
このときガウス過程モデルは,テストデータ$\bs{\xi} \in \mathbb{R}^M$の予測値の期待値を
\begin{equation}
    m(\bs{\xi}) = \widehat{m}(\bs{\xi}) + \left( \bs{y} - \widehat{\bs{m}} \right)\tran
      \left( \bs{K} + \sigma^2 \bs{I} \right)^{-1} \bs{k}(\bs{\xi}),
    \label{eqn:gp_exp}
\end{equation}
で与え,さらにテストデータ$\bs{\xi}_1$, $\bs{\xi}_2$の予測値の共分散を
\begin{equation}
    v(\bs{\xi}_1, \bs{\xi}_2) = k(\bs{\xi}_1, \bs{\xi}_2)
    - \bs{k}(\bs{\xi}_1)\tran \left( \bs{K} - \sigma^2 \bs{I} \right)^{-1} \bs{k}(\bs{\xi}_2),
    \label{eqn:gp_cov}
\end{equation}
で与えます.ただし関数$k: \mathbb{R}^M \times \mathbb{R}^M \to \mathbb{R}$はカーネル関数,
行列$\bs{K} \in \mathbb{R}^{N \times N}$は
\begin{equation}
    \bs{K} = \begin{pmatrix}
        k(\bs{x}_1, \bs{x}_1) & \cdots & k(\bs{x}_1, \bs{x}_N) \\
        \vdots                & \ddots & \vdots                \\
        k(\bs{x}_N, \bs{x}_1) & \cdots & k(\bs{x}_N, \bs{x}_N) \\
    \end{pmatrix},
\end{equation}
で与えられる行列,ベクトル$\bs{k}(\bs{\xi}) \in \mathbb{R}^N$は
\begin{equation}
    \bs{k}(\bs{\xi}) = \begin{pmatrix}
        k(\bs{\xi}, \bs{x}_1) \\
        \vdots                \\
        k(\bs{\xi}, \bs{x}_N) \\
    \end{pmatrix},
\end{equation}
です.ベクトル$\bs{y} \in \mathbb{R}^N$は学習データのラベルをベクトル状に並べたものであり,
$\bs{y} = (y_1, y_2, \ldots, y_N)\tran$です.
また$\widehat{m}(\bs{\xi})$は予測値の事前分布であり,$\widehat{\bs{m}}$は学習データの予測値の事前分布を
ベクトル状に並べたものです.特に事前分布を設定する必要がない場合は$\widehat{m}(\cdot) 
= 0$, $\widehat{\bs{m}} = \bs{0}$とするのが一般的です.

テストデータ$\bs{\xi}$の予測値の分散を求めるためには,共分散を求める式 (\ref{eqn:gp_cov}) において
$\bs{\xi}_1 =\bs{\xi}_2 = \bs{\xi}$とすれば良く,
\begin{equation}
    v(\bs{\xi}, \bs{\xi}) = k(\bs{\xi}, \bs{\xi})
    - \bs{k}(\bs{\xi})\tran \left( \bs{K} - \sigma^2 \bs{I} \right)^{-1} \bs{k}(\bs{\xi}),
\end{equation}
となります.

%%%%%%%%%%%%%%%%%%%%%%%%%%%%%%%%%%%%%%%%%%%%%%%%%%%%% SOURCE FINISH %%%%%%%%%%%%%%%%%%%%%%%%%%%%%%%%%%%%%%%%%%%%%%%%%%%%
% vim: expandtab tabstop=4 shiftwidth=4 fdm=marker


\section{RFF 再訪}
\titlebar
% TeX source
%
% Author: Tetsuya Ishikawa <tiskw111@gmail.com>
% Date  : October 13, 2021
%%%%%%%%%%%%%%%%%%%%%%%%%%%%%%%%%%%%%%%%%%%%%%%%%%%%% SOURCE START %%%%%%%%%%%%%%%%%%%%%%%%%%%%%%%%%%%%%%%%%%%%%%%%%%%%%

本節ではRFFの概要について述べます.
こちらも残念ながら詳細について述べる紙面的余裕がありませんので,
詳細をお知りになりたい場合は原論文 \cite{Rahimi2007} をご参照下さい.

関数$k: \mathbb{R}^M \times \mathbb{R}^M \to \mathbb{R}$をカーネル関数とします.
カーネル関数$k$をFourier変換することで
\begin{equation}
    k(\bs{x}_1, \bs{x}_2) \simeq \bs{\phi}({\bs{x}_1})\tran \bs{\phi}({\bs{x}_2}),
    \label{eqn:rff_kernel_approx}
\end{equation}
という近似式を求めるのがRFFです.このとき右辺のベクトル$\bs{\phi}(\bs{x}_1)$の次元$D$は任意に設定することができ,
次元$D$が大きいほど式 (\ref{eqn:rff_kernel_approx}) の近似精度が高い一方で,次元$D$を大きくすると計算量が多くなります.

具体例をひとつ挙げておきましょう.カーネル関数のひとつとして有名なRBFカーネルは
\begin{equation}
    k(\bs{x}_1, \bs{x}_2) = \exp \left( - \gamma \| \bs{x}_1 - \bs{x}_2 \|^2 \right),
\end{equation}
と与えられます.このRBFカーネルに対してRFFを適用すると
\begin{equation}
    \bs{\phi}(\bs{x}) = \begin{pmatrix}
        \cos \bs{Wx} \\
        \sin \bs{Wx}
    \end{pmatrix},
\end{equation}
となります.ただし行列$\bs{W} \in \mathbb{R}^{D/2 \times M}$は,
各要素を正規分布$\mathcal{N}(0, \frac{1}{4 \gamma})$にしたがってサンプリングしたランダム行列です.

%%%%%%%%%%%%%%%%%%%%%%%%%%%%%%%%%%%%%%%%%%%%%%%%%%%%% SOURCE FINISH %%%%%%%%%%%%%%%%%%%%%%%%%%%%%%%%%%%%%%%%%%%%%%%%%%%%
% vim: expandtab tabstop=4 shiftwidth=4 fdm=marker


\section{ガウス過程モデルと RFF}
\titlebar
% TeX source
%
% Author: Tetsuya Ishikawa <tiskw111@gmail.com>
% Date  : October 13, 2021
%%%%%%%%%%%%%%%%%%%%%%%%%%%%%%%%%%%%%%%%%%%%%%%%%%%%% SOURCE START %%%%%%%%%%%%%%%%%%%%%%%%%%%%%%%%%%%%%%%%%%%%%%%%%%%%%

本節ではガウス過程モデルにRFFを適用し,高速化の効果を理論的に確認します.

\subsection{RFF適用前のガウス過程モデルの計算量}

まずは通常のガウス過程モデルの学習および推論に要する計算量を確認しておきましょう.
前提として,入力ベクトルの次元$M$よりも学習データ数$N$の方が十分に大きいと仮定します.
このとき式 (\ref{eqn:gp_exp}) および (\ref{eqn:gp_cov}) のうち,学習時間のボトルネックは
明らかに逆行列$\left( \bs{K} + \sigma^2 \bs{I} \right)^{-1}$の計算にあります.
この行列の大きさは$N \times N$ですので,学習に要する計算量は$O(N^3)$となります.
次に推論ですが,テスト時のボトルネックは行列積
$\left( \bs{y} - \widehat{\bs{m}} \right)\tran \left( \bs{K} + \sigma^2 \bs{I} \right)^{-1}$
あるいは
$\bs{k}(\bs{\xi}_1)\tran \left( \bs{K} - \sigma^2 \bs{I} \right)^{-1} \bs{k}(\bs{\xi}_2)$
であり,これらの行列積に要する計算量はいずれも$O(N)$となります.

\subsection{予測値の期待値へのRFFの適用}

さて,ではいよいよガウス過程モデルにRFFを適用します.
まずは予測値の期待値ですが,式 (\ref{eqn:gp_exp}) にRFFの近似式 (\ref{eqn:gp_cov}) を代入すると
\begin{equation}
    m(\bs{\xi}) = \widehat{m}(\bs{\xi}) + \left( \bs{y} - \widehat{\bs{m}} \right)\tran
    \left( \bs{\Phi}\tran \bs{\Phi} + \sigma^2 \bs{I} \right)^{-1} \bs{\Phi}\tran \bs{\phi}(\bs{\xi}),
    \label{eqn:rffgp_exp_naive}
\end{equation}
となります.ただし行列$\bs{\Phi}$はRFFによって得られるベクトル$\bs{\phi}$を
学習データ全てに対して並べた$D \times N$行列
$\bs{\Phi} = (\bs{\phi}(\bs{x}_1), \ldots, \bs{\phi}(\bs{x}_N))$
です.しかし,まだこれでは高速化は図れていません.
式 (\ref{eqn:rffgp_exp_naive}) の計算量のボトルネックは依然として$N \times N$行列の逆行列のままです.

ここで一工夫します.式 (\ref{eqn:rffgp_exp_naive}) に対して逆行列の反転補題 (\textit{binominal inverse lemma}) 
を適用することを考えます.逆行列の反転補題とは以下の定理です.

\begin{theorem}[逆行列の反転補題]
    行列
    $\bs{A} \in \mathbb{R^{N \times N}}$,
    $\bs{B} \in \mathbb{R^{N \times M}}$,
    $\bs{C} \in \mathbb{R^{M \times N}}$,
    $\bs{D} \in \mathbb{R^{M \times M}}$
    に対して以下が成り立つ.
    \begin{align}
        &\left( \bs{A} + \bs{BDC} \right)^{-1} \notag \\
        &\hspace{10pt}
        = \bs{A}^{-1} - \bs{A}^{-1} \bs{B} \left( \bs{D}^{-1} + \bs{CA}^{-1} \bs{B} \right)^{-1} \bs{CA}^{-1}
        \label{eqn:bunominal_inverse_lemma}
    \end{align}
    ただし行列$\bs{A}$, $\bs{D}$は正則行列とする.
\end{theorem}
証明は本文書の末尾で行うものとし,ここではガウス過程モデルへのRFFの適用の話を先に進めさせて下さい.
上記の補題に対して
$\bs{A} = \sigma^2 \bs{I}$,
$\bs{B} = \bs{\Phi}\tran$,
$\bs{C} = \bs{\Phi}$,
$\bs{D} = \bs{I}$
とおけば,
\begin{equation*}
    \left( \bs{\Phi}\tran \bs{\Phi} + \sigma^2 \bs{I} \right)^{-1}
    = \frac{1}{\sigma^2} \left( \bs{I} - \bs{\Phi}\tran \left( 
    \bs{\Phi\Phi}\tran + \sigma^2 \bs{I} \right)^{-1} \bs{\Phi} \right),
\end{equation*}
を得ます.ここでさらに$\bs{P} = \bs{\Phi\Phi}\tran \in \mathbb{R}^{D \times D}$とおき,
上式の両辺に右から$\bs{\Phi}$をかけると
\begin{equation}
    \left( \bs{\Phi}\tran \bs{\Phi} + \sigma^2 \bs{I} \right)^{-1} \bs{\Phi}
    = \frac{1}{\sigma^2} \bs{\Phi}\tran \left(
    \bs{I} - \left( \bs{P} + \sigma^2 \bs{I} \right)^{-1} \bs{P} \right),
    \label{eqn:rff_key_eqn}
\end{equation}
を得ます.したがって式 (\ref{eqn:rffgp_exp_naive}) は
\begin{equation}
    m(\bs{\xi}) = \widehat{m}(\bs{\xi}) + \frac{1}{\sigma^2}
    \left( \bs{y} - \widehat{\bs{m}} \right)\tran \bs{\Phi}\tran \bs{S},
    \label{eqn:rffgp_exp}
\end{equation}
と書き改めることができます.ただし
\begin{equation}
    \bs{S} = \bs{I} - \left( \bs{P} + \sigma^2 \bs{I} \right)^{-1} \bs{P},
    \label{eqn:rffgp_exp_cache}
\end{equation}
です.

聡明なる読者は,すでにボトルネックが解消されていることにお気付きのことと思います.
式 (\ref{eqn:rffgp_exp_naive}) のボトルネックであった逆行列$( \bs{K} + \sigma^2 \bs{I})^{-1}$は,
式 (\ref{eqn:rffgp_exp}), (\ref{eqn:rffgp_exp_cache}) では $( \bs{P} + \sigma^2 \bs{I})^{-1}$となり,
行列のサイズは$D \times D$になりました.通常,RFFの次元$D$は学習データの総数$N$よりも十分小さく設定しますので,
もはやこの逆行列の計算はボトルネックではなくなりました.
式 (\ref{eqn:rffgp_exp}), (\ref{eqn:rffgp_exp_cache}) のボトルネックは行列積
$\bs{P} = \bs{\Phi\Phi}\tran$であり,この計算量は$O(ND^2)$です.
元のガウス過程モデルの学習に要する計算量が$O(N^3)$であったことを振り返れば,
RFFによってかなりの高速化が達成できたことになります.

\subsection{予測値の共分散へのRFFの適用}

次に予測値の共分散 (\ref{eqn:gp_cov}) にRFFを適用していきましょう.
式 (\ref{eqn:gp_cov}) に対してRFFの近似式 (\ref{eqn:gp_cov}) を代入し,さらに
式 (\ref{eqn:rff_key_eqn}) を適用すると
\begin{align}
    v(\bs{\xi}_1, \bs{\xi}_2)
    &= \bs{\phi}(\bs{\xi}_1)\tran \bs{\phi}(\bs{\xi}_2)
    - \frac{1}{\sigma^2} \bs{\phi}(\bs{\xi}_1)\tran \bs{PS} \bs{\phi}(\bs{\xi}_2)
    \notag \\
    &= \bs{\phi}(\bs{\xi}_1)\tran
    \left( \bs{I} - \frac{1}{\sigma^2} \bs{PS} \right)
    \bs{\phi}(\bs{\xi}_2),
    \label{eqn:rffgp_cov}
\end{align}
となります.式 (\ref{eqn:rffgp_cov}) のボトルネックは,
予測値の期待値と同様に行列積$\bs{P} = \bs{\Phi\Phi}\tran$であり,この計算量は$O(ND^2)$です.

ここで,RFFを適用した後のガウス過程モデルの学習および推論の手順を
疑似コードとしてAlgorithm \ref{alg:rffgp_train}, \ref{alg:rffgp_infer}にまとめておきます.
ただしAlgorithm \ref{alg:rffgp_train}, \ref{alg:rffgp_infer}では簡単のために事前分布を0としています.

\begin{algorithm}[t]
    \caption{\textgt{\bf RFF適用後のガウス過程モデルの学習}}
    \label{alg:rffgp_train}
    \KwData{$\mathcal{D} = \left\{ (\bs{x}_n, y_n) \right\}_{n=1}^{N}$, \, $\sigma \in \mathbb{R}^{+}$}
    \KwResult{$\bs{c}_\mathrm{m} \in \mathbb{R}^D$, \, $\bs{C}_\mathrm{v} \in \mathbb{R}^{D \times D}$}
    $\bs{y} \gets (y_1, \ldots, y_N)\tran$ \\
    $\bs{\Phi} \gets (\bs{\phi}(\bs{x}_1), \ldots, \bs{\phi}(\bs{x}_N))$ \\
    $\bs{P} \gets \bs{\Phi\Phi}\tran$ \\
    $\bs{S} \gets \bs{I} - \left( \bs{P} + \sigma^2 \bs{I} \right)^{-1} \bs{P}$ \\
    $\bs{c}_\mathrm{m} \gets \frac{1}{\sigma^2} \bs{y}\tran \bs{\Phi}\tran \bs{S}$
    \hfill\Comment{\textgt{\footnotesize 予測値の期待値の算出に使用}\hspace*{-36pt}\mbox{}}
    $\bs{C}_\mathrm{v} \gets \bs{I} - \frac{1}{\sigma^2} \bs{PS}$
    \hfill\Comment{\textgt{\footnotesize 予測値の共分散の算出に使用}\hspace*{-33pt}\mbox{}}
\end{algorithm}

\begin{algorithm}[t]
    \caption{\textgt{\bf RFF適用後のガウス過程モデルの推論}}
    \label{alg:rffgp_infer}
    \KwData{$\bs{\xi} \in \mathbb{R}^M$, \, $\bs{c}_\mathrm{m} \in \mathbb{R}^D$, \, $\bs{C}_\mathrm{v} \in \mathbb{R}^{D \times D}$}
    \KwResult{$\mu \in \mathbb{R}$, $\eta \in \mathbb{R}$}
    $\bs{z} \gets \bs{\phi}(\bs{\xi})$ \\
    $\mu \gets \bs{c}_\mathrm{m} \bs{z}$
    \hfill\Comment{\textgt{\footnotesize 予測値の期待値の算出}\hspace*{-80pt}\mbox{}}
    $\eta \gets \bs{z}\tran \bs{C}_\mathrm{v} \bs{z}$
    \hfill\Comment{\textgt{\footnotesize 予測値の共分散の算出}\hspace*{-68pt}\mbox{}}
\end{algorithm}

最後に,RFFを適用した後の計算量を表\ref{tab:gp_complexity}に整理しました.
ただし$N \in \mathbb{Z}^+$は学習データの総数,$D \in \mathbb{Z}^+$はRFFの次元です.

\begin{table}[t]
    \caption{RFF適用前後でのガウス過程モデルの計算量}
    \label{tab:gp_complexity}
    \begin{center}\begin{tabular}{ccc}
        \hline
         & 学習 & 推論 \\
        \hline
        RFF適用前 & $O(N^3)$   & $O(N)$    \\  
        RFF適用後 & $O(N D^2)$ & $O(D^2)$  \\
        \hline
    \end{tabular}\end{center}
\end{table}

%%%%%%%%%%%%%%%%%%%%%%%%%%%%%%%%%%%%%%%%%%%%%%%%%%%%% SOURCE FINISH %%%%%%%%%%%%%%%%%%%%%%%%%%%%%%%%%%%%%%%%%%%%%%%%%%%%
% vim: expandtab tabstop=4 shiftwidth=4 fdm=marker


% \section{より天下り的でない導出}
% \titlebar
% % TeX source
%
% Author: Tetsuya Ishikawa <tiskw111@gmail.com>
% Date  : October 13, 2021
%%%%%%%%%%%%%%%%%%%%%%%%%%%%%%%%%%%%%%%%%%%%%%%%%%%%% SOURCE START %%%%%%%%%%%%%%%%%%%%%%%%%%%%%%%%%%%%%%%%%%%%%%%%%%%%%

前節の説明は,簡潔ではありますが,やや天下り的なきらいがあります.
本節では,もう少し丁寧にご説明をしようと思います.

まずはガウス過程モデルの定式化と記号の導入から始めます.

%%%%%%%%%%%%%%%%%%%%%%%%%%%%%%%%%%%%%%%%%%%%%%%%%%%%% SOURCE FINISH %%%%%%%%%%%%%%%%%%%%%%%%%%%%%%%%%%%%%%%%%%%%%%%%%%%%
% vim: expandtab tabstop=4 shiftwidth=4 fdm=marker


\appendix

\section{補足}
\titlebar
% TeX source
%
% Author: Tetsuya Ishikawa <tiskw111@gmail.com>


\subsection{逆行列の反転補題の証明}

逆行列の反転補題を再掲し,証明します.

\begin{theorem}[逆行列の反転補題]
    行列
    $\bs{A} \in \mathbb{R^{N \times N}}$,
    $\bs{B} \in \mathbb{R^{N \times M}}$,
    $\bs{C} \in \mathbb{R^{M \times N}}$,
    $\bs{D} \in \mathbb{R^{M \times M}}$
    に対して以下が成り立つ.
    \begin{align*}
        &\left( \bs{A} + \bs{BDC} \right)^{-1} \\
        &\hspace{10pt}
        = \bs{A}^{-1} - \bs{A}^{-1} \bs{B} \left( \bs{D}^{-1} + \bs{CA}^{-1} \bs{B} \right)^{-1} \bs{CA}^{-1}
    \end{align*}
    ただし行列$\bs{A}$, $\bs{D}$は正則行列とする.
\end{theorem}
\begin{proof}
以下の等式が成立する.
\begin{align*}
    \begin{pmatrix}
        \bs{A} & \bs{B} \\
        \bs{C} & \bs{D}
    \end{pmatrix}^{-1}
    &= \begin{pmatrix}
        \bs{A}^{-1} + \bs{A}^{-1}\bs{BSCA}^{-1} & - \bs{A}^{-1}\bs{BS} \\
        - \bs{SCA}^{-1}                         & \bs{S}
    \end{pmatrix} \\
    &= \begin{pmatrix}
        \bs{T}                & - \bs{TBD}^{-1} \\
        - \bs{D}^{-1} \bs{CT} & \bs{D}^{-1} + \bs{D}^{-1} \bs{CTBD}^{-1}
    \end{pmatrix},
\end{align*}
ただし
\begin{align}
    \bs{T} &= \left( \bs{D} - \bs{CA}^{-1} \bs{B} \right)^{-1}, \\
    \bs{S} &= \left( \bs{A} - \bs{BD}^{-1} \bs{C} \right)^{-1},
\end{align}
とする.これは直接計算により明らか.さて,上記ブロック行列の対応する箇所を比較することで
\begin{align}
    \bs{T} &= \bs{A}^{-1} + \bs{A}^{-1}\bs{BSCA}^{-1},
    \label{eqn:binominal_theorem_proof_1} \\
    \bs{S} &= \bs{D}^{-1} + \bs{D}^{-1} \bs{CTBD}^{-1}, \\
    - \bs{A}^{-1} \bs{BS} &= - \bs{TBD}^{-1}, \\
    - \bs{SCA}^{-1} &= - \bs{D}^{-1} \bs{CT},
\end{align}
を得る.式 (\ref{eqn:binominal_theorem_proof_1}) に対して
\begin{center}
    $\bs{A} \to \bs{D}^{-1}$, \hspace{5pt}
    $\bs{B} \to - \bs{C}$, \hspace{5pt}
    $\bs{C} \to \bs{B}$, \hspace{5pt}
    $\bs{D} \to \bs{A}$,
\end{center}
と置き直せば,証明すべき式を得る.
\end{proof}


% vim: expandtab tabstop=4 shiftwidth=4 fdm=marker


\section*{おわりに}
\titlebar
% TeX source
%
% Author: Tetsuya Ishikawa <tiskw111@gmail.com>


TBD


% 最後に,私の数学的活動は,2017年に逝去された恩師,山下弘一郎先生や,
% 大学および大学院で私の指導教官を担当して下さった早川朋久准教授をはじめ,
% 数学で私と関わりを持ったすべての方々のおかげで成り立っています.
% そして,数学的活動の以前に,そもそも私の生は両親によって与えられています.


% vim: expandtab tabstop=4 shiftwidth=4 fdm=marker


% \pagebreak

% TeX source
%
% Author: Tetsuya Ishikawa <tiskw111@gmail.com>
% Date  : October 13, 2021
%%%%%%%%%%%%%%%%%%%%%%%%%%%%%%%%%%%% SOURCE START %%%%%%%%%%%%%%%%%%%%%%%%%%%%%%%%%%%

\begin{thebibliography}{9}

    \bibitem{Rahimi2007}
    A.~Rahimi and B.~Recht, 
    ``Random Features for Large-Scale Kernel Machines'',
    Neural Information Processing Systems, 2007.

    \bibitem{Rasmussen2006}
    C.~Rasmussen and C.~Williams, ``Gaussian Processes for Machine Learning'', MIT Press, 2006.

    \bibitem{Akaho2018}
    赤穂昭太郎, ``ガウス過程回帰の基礎'', システム/制御/情報, vol. 62, no. 10, pp. 390-395, 2018.
    {\footnotesize\tt https://www.jstage.jst.go.jp/article/isciesci/62/10/62\_390/\_pdf}

    \bibitem{rfflearn} {\tt https://github.com/tiskw/random-fourier-features}

\end{thebibliography}

%%%%%%%%%%%%%%%%%%%%%%%%%%%%%%%%%%%% SOURCE FINISH %%%%%%%%%%%%%%%%%%%%%%%%%%%%%%%%%%
% vim: expandtab shiftwidth=4 tabstop=4 filetype=tex


\end{document}


%%%%%%%%%%%%%%%%%%%%%%%%%%%%%%%%%%%%%%%%%%%%%%%%%%%%% SOURCE FINISH %%%%%%%%%%%%%%%%%%%%%%%%%%%%%%%%%%%%%%%%%%%%%%%%%%%%
% vim: expandtab tabstop=4 shiftwidth=4 fdm=marker
