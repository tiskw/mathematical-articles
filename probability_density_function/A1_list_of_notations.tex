% TeX source
%
% Author: Tetsuya Ishikawa <tiskw111@gmail.com>


本節では,本文書で用いられている数学記号を簡潔にまとめておく.

\begin{table*}[b]
\begin{center}
\begin{tabular}{p{30pt}p{350pt}p{100pt}}
    \hline
    記号 & 意味 & 使用例 \\
    \hline
    $\triangleq$ &
    記号 $\triangleq$の左辺を右辺で定義する.等号$=$に近い意味合いを持つが,
    変数や関数の定義を行う際などに,それが「定義だ」ということを明示したいときに使用する.
    記号 $:=$ を用いることもある. &
    $f(x) \triangleq x^2$ \\
    $\frac{\mathrm{d} f}{\mathrm{d} x}$ &
    関数 $f$ の変数 $x$ による微分.&
    $\frac{\mathrm{d} f}{\mathrm{d} x} = 2 x$ \\
    $\frac{\partial f}{\partial x}$ &
    関数 $f$ の変数 $x$ による偏微分.&
    $\frac{\partial f}{\partial x} = 2 x$ \\
    \hline
\end{tabular}
\end{center}
\end{table*}


% vim: expandtab tabstop=4 shiftwidth=4 fdm=marker
