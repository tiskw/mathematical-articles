% TeX source


そもそも,なぜ確率密度関数が必要なのだろうか.実は目的は非常にシンプルであり
「ある確率変数 $X$ が特定の値 $x$ をとる確率を表現したい」というだけである.
こんな簡単なことが,実は連続値を取る確率変数では表現が難しいのである.

有限個の値しか取らない確率変数ではこのような困難は生じない.
まずはそれを確認するところから始めよう.
有限個の値しか取らない確率変数として,正しく作られたサイコロの出る目を表す
確率変数 $D$ を考えてみよう.確率変数 $D$ は $1, 2, 3, 4, 5, 6$ の 6 通りの
値を取り,その確率はそれぞれ
\begin{equation}
    P(D = 1) = P(D = 2) = \cdots = P(D = 6) = \frac{1}{6},
\end{equation}
である.確率変数 $D$ がどの値をとるにしろ,その確率は上記の通り簡単に表現ができる.
何の問題もない.

しかしこれが連続値を取る確率変数では簡単ではない.
連続値を取る確率変数として,ここでは区間 $[0, 1]$ 上で一様分布にしたがう実数値の確率変数 $X$
を考えてみよう.区間 $[0, 1]$ には無限個の実数が含まれていることに注意されたい.
さて,確率変数 $X$ は一様分布にしたがうのであるから,
\begin{equation}
    P(X = 0) = \cdots = P(X = 0.5) = \cdots = P(X = 1),
\end{equation}
である.しかし確率変数 $X$ の取りうる値は無限個あるのに対し,確率の総和は 1 であるから,
\begin{equation}
    P(X = 0) = \cdots = P(X = 0.5) = \cdots = P(X = 1) = 0,
    \label{eqn:uniform_dist_on_01}
\end{equation}
とならざるをえない(この確率が 0 になってしまうことの厳密な説明は次章に行うので,
今は直感的に理解ができれば十分である).しかしこれは明らかに表現として不十分である.
式 (\ref{eqn:uniform_dist_on_01}) は決して間違えてはいない.しかし式 (\ref{eqn:uniform_dist_on_01}) を
見るとあたかも確率変数はどの値も取らないように見えてしまう.
しかし実際にそんなことはなく,確率変数 $X$ は区間 $[0, 1]$ 上の一様分布なのである.

つまり,連続値を取る確率変数 $X$ に対しては,$X = x$ となる確率を定義するのに通常の $P(X = x)$ という
表現は,結果が常に 0 になってしまうので,適さないのである.ではどうすれば良いのだろうか.

この困難を解決してくれるのが確率密度関数なのである.
次章以降でこれを説明していこう.


% vim: expandtab tabstop=4 shiftwidth=4 fdm=marker
