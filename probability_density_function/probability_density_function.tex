% TeX source


\documentclass[uplatex, twocolumn, a4paper, 10pt, dvipdfmx]{jsarticle}

\usepackage[dvipdfm,width=190truemm,height=270truemm]{geometry}
\usepackage{amsmath,amssymb}
\usepackage{lmodern}
\usepackage{helvet}
\usepackage[ipaex, unicode]{pxchfon}
\usepackage[dvipdfmx]{graphicx}
\usepackage{thmbox}
\usepackage{csquotes}
\usepackage{algorithm2e}
\usepackage{algorithmic}
\usepackage{fancyhdr}
\usepackage{xcolor}

\pagestyle{fancy}
\lhead{}
\rhead{}
\cfoot{-- {\thepage} --}
\rfoot{\small\tt\textcolor{lightgray}{T.~Ishikawa}}
\renewcommand{\headrulewidth}{0pt}
\renewcommand{\footrulewidth}{0pt}

\allowdisplaybreaks

\newcommand{\bs}[1]{\boldsymbol{#1}}
\newcommand{\mbar}{\hspace{1pt}\bigm|\hspace{1pt}}
\newcommand{\tran}{^{\mkern-1.5mu\mathsf{T}}}
\DeclareMathOperator*{\sgn}{sgn}
\DeclareMathOperator*{\Sgn}{Sgn}
\DeclareMathOperator*{\argmin}{argmin}

\newtheorem[M]{theorem}{定理}[section]
\newtheorem[M]{definition}[theorem]{定義}
\newtheorem[M]{lemma}[theorem]{補題}

\renewenvironment{proof}%
{\textsf{証明}:}{\hfill$\blacksquare$\vspace*{8pt}}

\RestyleAlgo{ruled}
\SetKwComment{Comment}{/* }{ */}

\newcommand{\titlebar}{
\setlength\unitlength{1truemm}
\vspace*{-19pt}
\begin{picture}(88,0)(0,0)
\put(-4,0){\line(1,0){93}}
\put(89,0){\circle*{1}}
\end{picture}}

\begin{document}

\title{\textsf{\bfseries 確率密度関数の定義再訪}}
\author{石川 徹也 \quad \texttt{<tiskw111@gmail.com>}}
\date{2022年6月26日}
\maketitle
\thispagestyle{fancy}

\section*{はじめに}
\titlebar
% TeX source
%
% Author: Tetsuya Ishikawa <tiskw111@gmail.com>
% Date  : October 13, 2021
%%%%%%%%%%%%%%%%%%%%%%%%%%%%%%%%%%%%%%%%%%%%%%%%%%%%% SOURCE START %%%%%%%%%%%%%%%%%%%%%%%%%%%%%%%%%%%%%%%%%%%%%%%%%%%%%

本文書では,ガウス過程モデル~\cite{Rahimi2007}に
random Fourier features~\cite{Rasmussen2006} を適用する手順を解説します.
これにより,ガウス過程モデルの学習や推論を高速化させることができ,
より大規模なデータにガウス過程モデルを適用することができます.

ガウス過程モデル~\cite{Rahimi2007}は確率的教師あり機械学習フレームワークのひとつであり,
サポートベクトルマシンやランダムフォレストなどと並んで,回帰や分類タスクに広く使用されています.
ガウス過程モデルがサポートベクトルマシンやランダムフォレストと大きく違う点は「確率的なモデルである」ことです.
すなわち,ガウス過程モデルは確率的なモデルとして定式化されているため,予測値だけでなく,
その予測に対する不確実性の尺度をも提供することができます.
これは機械学習の説明性を上げることのできる非常に有益な性質です.

その一方で,ガウス過程モデルは学習や推論の計算コストが高いことでも知られています.
学習データの総数を$N \in \mathbb{Z}^{+}$としたとき,
ガウス過程モデルの学習に要する計算量は$O(N^3)$,推論に要する計算量は$O(N^2)$です.
問題は計算量が学習データの総数$N$のべき乗になってしまっていることで,これは大規模データへの適用に際して障害になり得ます.
これはガウス過程モデルがカーネル法と同等の数学的構造を有していることに起因しており,
言い変えれば,カーネルサポートベクトルマシンも同じ悩みを有しています.

カーネル法を高速化する手法のひとつにrandom Fourier features~\cite{Rasmussen2006}があります(以下 RFF と略します).
これはカーネル関数を有限次元ベクトルの内積として近似することで,
カーネル法の柔軟性を維持しつつ計算量を大幅に削減する手法です.
具体的には学習に要する計算量を$O(N D^2)$,推論に要する計算量を$O(D^2)$にまで削減することができます.
ただし$D \in \mathbb{Z}^{+}$は RFF のハイパーパラメータであり,
学習データの総数$N$とは独立に指定することができます.

ガウス過程モデルはカーネル法と同等の数学的構造を有しているため,ガウス過程モデルにも RFF を適用することができます.
これにより,ガウス過程モデルはより強力かつお手軽に使える,非常に頼もしいツールへと進化します.


しかしながら,ガウス過程モデルへのRFFの適用にあたっては,実は一筋縄ではいかないところがあります.
ただRFFを適用するだけでは高速化につながらず,一工夫する必要があるのです.
ですが,そのあたりの困難さや解決方法に言及している文献が,残念ながら世の中には存在しないようでしたので,
本文書にてその手順を解説しようと考えた次第です.

ちなみに,本文書は拙作ライブラリ\texttt{rfflearn}~\cite{rfflearn}に同梱されているドキュメントの日本語版です.
おそらく\texttt{rfflearn}に同梱されている文書の方が頻繁にメンテナンスされると思いますので,
英語でも差し支えない方はそちらをご参照下さい.

\begin{displayquote}\footnotesize\textsf{NOTE:}
    上述のライブラリ \texttt{rfflearn} は以下で公開されています.
    \begin{center}
        \texttt{https://github.com/tiskw/random-fourier-features}
    \end{center}
\end{displayquote}

%%%%%%%%%%%%%%%%%%%%%%%%%%%%%%%%%%%%%%%%%%%%%%%%%%%%% SOURCE FINISH %%%%%%%%%%%%%%%%%%%%%%%%%%%%%%%%%%%%%%%%%%%%%%%%%%%%
% vim: expandtab tabstop=4 shiftwidth=4 fdm=marker


\section{古典的確率論の破綻}
\titlebar
% TeX source


そもそも,なぜ確率密度関数が必要なのだろうか.実は目的は非常にシンプルであり
「ある確率変数 $X$ が特定の値 $x$ をとる確率を表現したい」というだけである.
こんな簡単なことが,実は連続値を取る確率変数では表現が難しいのである.

有限個の値しか取らない確率変数ではこのような困難は生じない.
まずはそれを確認するところから始めよう.
有限個の値しか取らない確率変数として,正しく作られたサイコロの出る目を表す
確率変数 $D$ を考えてみよう.確率変数 $D$ は $1, 2, 3, 4, 5, 6$ の 6 通りの
値を取り,その確率はそれぞれ
\begin{equation}
    P(D = 1) = P(D = 2) = \cdots = P(D = 6) = \frac{1}{6},
\end{equation}
である.確率変数 $D$ がどの値をとるにしろ,その確率は上記の通り簡単に表現ができる.
何の問題もない.

しかしこれが連続値を取る確率変数では簡単ではない.
連続値を取る確率変数として,ここでは区間 $[0, 1]$ 上で一様分布にしたがう実数値の確率変数 $X$
を考えてみよう.区間 $[0, 1]$ には無限個の実数が含まれていることに注意されたい.
さて,確率変数 $X$ は一様分布にしたがうのであるから,
\begin{equation}
    P(X = 0) = \cdots = P(X = 0.5) = \cdots = P(X = 1),
\end{equation}
である.しかし確率変数 $X$ の取りうる値は無限個あるのに対し,確率の総和は 1 であるから,
\begin{equation}
    P(X = 0) = \cdots = P(X = 0.5) = \cdots = P(X = 1) = 0,
    \label{eqn:uniform_dist_on_01}
\end{equation}
とならざるをえない(この確率が 0 になってしまうことの厳密な説明は次章に行うので,
今は直感的に理解ができれば十分である).しかしこれは明らかに表現として不十分である.
式 (\ref{eqn:uniform_dist_on_01}) は決して間違えてはいない.しかし式 (\ref{eqn:uniform_dist_on_01}) を
見るとあたかも確率変数はどの値も取らないように見えてしまう.
しかし実際にそんなことはなく,確率変数 $X$ は区間 $[0, 1]$ 上の一様分布なのである.

つまり,連続値を取る確率変数 $X$ に対しては,$X = x$ となる確率を定義するのに通常の $P(X = x)$ という
表現は,結果が常に 0 になってしまうので,適さないのである.ではどうすれば良いのだろうか.

この困難を解決してくれるのが確率密度関数なのである.
次章以降でこれを説明していこう.


% vim: expandtab tabstop=4 shiftwidth=4 fdm=marker


\section{Reimann積分論的確率論}
\titlebar
\input{02_reimann_probability_theory.tex}

\section{確率密度関数}
\titlebar
\input{03_probability_density_function.tex}

\section{累積分布関数との関係}
\titlebar
% TeX source


TBD


% vim: expandtab tabstop=4 shiftwidth=4 fdm=marker


\section*{おわりに}
\titlebar
% TeX source
%
% Author: Tetsuya Ishikawa <tiskw111@gmail.com>


TBD


% 最後に,私の数学的活動は,2017年に逝去された恩師,山下弘一郎先生や,
% 大学および大学院で私の指導教官を担当して下さった早川朋久准教授をはじめ,
% 数学で私と関わりを持ったすべての方々のおかげで成り立っています.
% そして,数学的活動の以前に,そもそも私の生は両親によって与えられています.


% vim: expandtab tabstop=4 shiftwidth=4 fdm=marker


% \pagebreak

% TeX source
%
% Author: Tetsuya Ishikawa <tiskw111@gmail.com>
% Date  : Aug  4, 2019
%%%%%%%%%%%%%%%%%%%%%%%%%%%%%%%%%%%% SOURCE START %%%%%%%%%%%%%%%%%%%%%%%%%%%%%%%%%%%


\begin{thebibliography}{9}

\bibitem{Vapnik1963}
V.~Vapnik and A.~Lerner,
``Pattern recognition using generalized portrait method'',
Automation and Remote Control, vol. 24, 1963.

\bibitem{Boser1992}
B.~Boser, I.~Guyon and V.~Vapnik,
``A training algorithm for optimal margin classifiers'',
Proc. of the fifth annual workshop on Computational learning theory, pp. 144-152, 1992

\bibitem{Akaho2008}
赤穂昭太郎,
「カーネル多変量解析」,
岩波書店,2008.

\bibitem{Jost2005}
J. Jost,
``Postmodern Analysis'',
Springer, 2005.

\bibitem{Horiuchi2005}
堀内利郎,下村勝孝,
「関数解析の基礎 -- $\infty $次元の微積分」,
内田老鶴圃,2005.

\bibitem{Bishop2010}
C. Bishop,
``Pattern Recognition and Machine Learning'',
Springer, 2010.

\bibitem{Rahimi2007}
A.~Rahimi and B.~Recht, 
``Random Features for Large-Scale Kernel Machines'',
Neural Information Processing Systems, 2007.

\bibitem{Yu2016}
F.~X.~Yu, A.~T.~Suresh, K.~Choromanski, D.~H.~Rice and S.~Kumar,
``Orthogonal Random Features'',
Neural Information Processing Systems, 2016.

\end{thebibliography}


%%%%%%%%%%%%%%%%%%%%%%%%%%%%%%%%%%%% SOURCE FINISH %%%%%%%%%%%%%%%%%%%%%%%%%%%%%%%%%%
% vim: expandtab shiftwidth=4 tabstop=4 filetype=tex


\end{document}


% vim: expandtab tabstop=4 shiftwidth=4 fdm=marker
